\documentclass{article}

\usepackage[english]{babel}
\usepackage{eurosym}
\usepackage{iflang}
\usepackage{resume}

\begin{document}

\selectlanguage{english}

\name{\bf EDGAR RIBA}

\begin{llist}
\label{Education}
\sectiontitle{Education}

\employer{UNIVERSITAT AUTONOMA DE BARCELONA} \location{Barcelona, Spain}
\dates{10/2015--now}
Ph.D. in Computer Science\\
Dissertation: \textit{Computer Vision Techniques in Very
High Resolution Remote Sensing}.\\
Advisor: Daniel Ponsa.\\
Research Topics: Structure from Motion, Multiview Geometry, Deep Learning, Drone Imagery, Remote Sensing, Multi/Hyper spectral imagery.

\employer{UNIVERSITAT POLITECNICA DE CATALUNYA} \location{Barcelona, Spain}
\dates{07/2012--06/2015}
M.S. in Automatic Control and Robotics.\\
Dissertation: \textit{Implementation of a 3D pose estimation algorithm}.\\
Advisors: Adrian Penate and Francesc Moreno.

\employer{UNIVERSITAT POLITECNICA DE CATALUNYA} \location{Barcelona, Spain}
\dates{07/2008--06/2012}
B.S. in Geomatic and Surveying Engineering.


% Research Experience
\label{Work Experience}
\sectiontitle{Work Experience}
\vspace{-0.33cm}

\employer{COMPUTER VISION CENTER}\location{Barcelona, Spain}
\dates{10/2015--now}
PhD candidate, Advanced Driver Assistance Systems group
\vspace{-0.33cm}
\begin{itemize}
 \item working on algorithms for 3D reconstruction and deep learning for drone imagery.
\end{itemize}

\employer{OPENCV}\location{Barcelona, Spain}
\dates{05/2015--08/2015}
Intern, Google Summer of Code
\vspace{-0.33cm}
\begin{itemize}
 \item developed the Structure From Motion module using a customized version of Libmv.
 \item integrated and succeed with the pull request in the Github infrastructure.
\end{itemize}

\employer{ALDEBARAN ROBOTICS}\location{Paris, France}
\dates{02/2015--06/2015}
Intern Software Engineer, Perception team
\vspace{-0.33cm}
\begin{itemize}
 \item designed and implemented an algorithm for people detection and tracking by sensor fusion using ROS, OpenCV and C++.
\end{itemize}

\employer{OPENCV}\location{Barcelona, Spain}
\dates{05/2014--08/2014}
Intern, Google Summer of Code
\vspace{-0.33cm}
\begin{itemize}
 \item designed and implemented a real time pose estimation algorithm for textured objects.
 \item implemented the PnP method: \textit{A direct least-squares} (DLS) in the calib3d module.
 \item contributed with a tutorial for the calib3d module.
\end{itemize}

\employer{INSTITUT DE ROBOTICA I INFORMATICA INDUSTRIAL (CSIC-UPC) }\location{Barcelona, Spain}
\dates{05/2014--08/2014}
Research Assistant, Perception and Manipulation group
\vspace{-0.33cm}
\begin{itemize}
 \item worked on my masters thesis in geometric computer vision.
\end{itemize}


% Teaching Experience
\label{Teaching Experience}
\sectiontitle{Teaching Experience}

\employer{UNIVERSITAT POLITECNICA DE CATALUNYA}\location{Barcelona, Spain}
310209, {\em Electromagnetism and Optics}, Fall 2007: Teaching Assistant\\
310209, {\em Electromagnetism and Optics}, Spring 2007: Teaching Assistant\\

% Professional Activities
\sectiontitle{Professional Activities}
\vspace{-0.4cm}

Co-founder and member of the \textit{La Konfraria de la Vila del Pingui}. An Open Source community, organizing local events and workshops spreading the free software culture.

Google Summer of Code student for OpenCV from 2014 to 2015.

Constant developer in many Open Source projects such as OpenCV, ROS, Object Recognition Kitchen, OpenDroneMap and OpenSfM.

Participated in many robotics competitions such as the HUMABOT Robot Competition 2014 during the \textit{IEEE-RAS International Conference on Humanoid Robots} (Madrid, Spain) and the RoCKIn Camp 2014 organized by  La Sapienza University of Rome (Rome, Italy) within the company PAL Robotics.


% Skills
\label{Skills}
\sectiontitle{Skills}
Competence: Vision (SfM, object recognition, people detection), Robotics, Programming.\\
Programming Languages: C++, Python.\\
Programming Libraries: OpenCV, ROS, Caffe, OpenSfM, CMVS/PMVS. \\
Extra Interests: CMake, Git, Android, Octave/Matlab, Java, Javascript, Boost. \\
Languages: Catalan (native), Spanish (native), English  (professional)


% Software
%\sectiontitle{Software}
%{\em ROS packages}: maintainer/developer of 60+ ROS (Robot Operating System) packages about computer vision, lasers, 
%graph processing, the Aldebaran robots.
%
%{\em Object Recognition Kitchen}: set of tools to develop and execute object recognition.


% Papers
%\sectiontitle{Journal Articles}
%\input{publications_journals.tex}

%\sectiontitle{Papers in Reviewed Proceedings}
%V.~Balntas, E.~Riba, D.~Ponsa, K.~Mikolajczyk, ``Learning local feature descriptors with triplets and shallow convolutional neural networks'', \textbf{BMVC}, 2016.


% Relevant Coursework
\label{Relevant Coursework}
\sectiontitle{Relevant Coursework}
- \textit{Object Detection} (Free online course, Universitat Autonoma de Barcelona. Fall 2015)\\
- \textit{Machine Learning} (Free online course, Stanford. Fall 2014)\\
- \textit{Introduction to Robot Operating System (ROS)} (Universitat Politecnica de Catalunya. Spring 2014)

\label{Reference}
\sectiontitle{References}
\textbf{Dr. Gary Bradski}\\
OSVF/OpenCV.org\\
garybradski@osvf.org

\textbf{Dr. Daniel Ponsa}\\
Computer Vision Center, Universitat Autonoma de Barcelona.\\
daniel@cvc.uab.es

\textbf{Dr. Francesc Moreno-Noguer}\\
Institut de Robotica i Informatica Industrial (CSIC-UPC)   
\\
fmoreno@iri.upc.edu

\textbf{Dr. Vincent Rabaud}\\
Google, Inc.\\
vincent.rabaud@gmail.com



\end{llist}

{\em Last update: \today}

\end{document}
