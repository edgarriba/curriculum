\documentclass{article}

\usepackage[english]{babel}
\usepackage{eurosym}
\usepackage{iflang}
\usepackage{resume}
\usepackage{hyperref}

\begin{document}

\selectlanguage{english}

\name{\bf EDGAR RIBA}

\begin{llist}
\label{Education}
\sectiontitle{Education}

\employer{UNIVERSITAT AUTONOMA DE BARCELONA} \location{Barcelona, Spain}
\dates{10/2015--now}
Ph.D. in Computer Science\\
Dissertation: \textit{Geometric Computer Vision using deep nets}.\\
Advisor: Daniel Ponsa.\\
Research Topics: Deep Learning, Local Features, Camera Pose and depth estimation.

\employer{UNIVERSITAT POLITECNICA DE CATALUNYA} \location{Barcelona, Spain}
\dates{07/2012--06/2015}
M.S. in Automatic Control and Robotics.\\
Dissertation: \textit{Implementation of a 3D pose estimation algorithm}.\\
Advisors: Adrian Penate and Francesc Moreno-Noguer.

\employer{UNIVERSITAT POLITECNICA DE CATALUNYA} \location{Barcelona, Spain}
\dates{07/2008--06/2012}
B.S. in Geomatic and Surveying Engineering.


% Research Experience
\label{Work Experience}
\sectiontitle{Work Experience}
\vspace{-0.33cm}

\employer{OSVF/OPENCV.org}\location{}
\dates{04/2018--now}
Member of the Board of the OpenCV library fundation
\vspace{-0.33cm}
\begin{itemize}
 \item representing the OpenCV library in public events to advertise the Open Computer Vision framework.
 \item organizing and mentoring collaboration projects (e.g. during Google Summer of Code).
\end{itemize}

\employer{ARRAIY, Inc.}\location{Barcelona, Spain.}
\dates{11/2017--06/2019}
PhD candidate and Research Engineering Intern
\vspace{-0.33cm}
\begin{itemize}
 \item leading the Open Source team of the company.
 \item maintainance and development of the Kornia open source project.
 \item remotely working on research for depth estimation with deep nets and 3D geometric computer vision for VFX content generation.
\end{itemize}

\employer{ARRAIY, Inc.}\location{Palo Alto, CA, USA.}
\dates{05/2017--10/2017}
Research Engineering Intern, Geometry group
\vspace{-0.33cm}
\begin{itemize}
 \item developed and integrated an internal framework for differential 3D geometry using deep nets.
\end{itemize}

\employer{COMPUTER VISION CENTER}\location{Barcelona, Spain}
\dates{10/2015--05/2017}
PhD candidate, MultiSpectral Image Analysis and Understanding group
\vspace{-0.33cm}
\begin{itemize}
 \item worked on deep learning applied to local features.
 \item holder of the trainee research staff grant in the CS department at the Universitat Autonoma de Barcelona.
\end{itemize}

\employer{OPENCV}\location{online}
\dates{05/2016--08/2016}
Intern, Google Summer of Code
\vspace{-0.33cm}
\begin{itemize}
 \item integrated tiny-dnn to OpenCV contrib by adding a wrapper to the caffe converter.
 \item fixed bugs in tiny-dnn and developed several new features such as GPU support via OpenCL and NNPACK optimizations.
\end{itemize}

\employer{OPENCV}\location{online}
\dates{05/2015--08/2015}
Intern, Google Summer of Code
\vspace{-0.33cm}
\begin{itemize}
 \item developed the Structure From Motion module using a customized version of Libmv.
\end{itemize}

\employer{ALDEBARAN ROBOTICS}\location{Paris, France}
\dates{02/2015--06/2015}
Intern Software Engineer, Perception team
\vspace{-0.33cm}
\begin{itemize}
 \item designed and implemented an algorithm for people detection and tracking by sensor fusion using ROS, OpenCV and C++.
\end{itemize}

\employer{OPENCV}\location{online}
\dates{05/2014--08/2014}
Intern, Google Summer of Code
\vspace{-0.33cm}
\begin{itemize}
 \item designed and implemented a real time pose estimation algorithm for textured objects.
 \item implemented the PnP method: \textit{A direct least-squares} (DLS) in the calib3d module.
 \item contributed with a tutorial for the calib3d module.
\end{itemize}

\employer{INSTITUT DE ROBOTICA I INFORMATICA INDUSTRIAL (CSIC-UPC) }\location{Barcelona, Spain}
\dates{05/2014--08/2014}
Research Assistant, Perception and Manipulation group
\vspace{-0.33cm}
\begin{itemize}
 \item worked on my masters thesis in geometric computer vision.
\end{itemize}


% Teaching Experience
\label{Teaching Experience}
\sectiontitle{Teaching Experience}

\employer{UNIVERSITAT AUTONOMA DE BARCELONA}\location{Barcelona, Spain}
102708, {\em Software Engineering Fundamentals}, Spring 2018: Associate Professor\\

\employer{UNIVERSITAT AUTONOMA DE BARCELONA}\location{Barcelona, Spain}
102708, {\em Software Engineering Fundamentals}, Spring 2017: Teaching Assistant\\

\employer{UNIVERSITAT POLITECNICA DE CATALUNYA}\location{Barcelona, Spain}
310209, {\em Electromagnetism and Optics}, Fall 2007: Teaching Assistant\\
310209, {\em Electromagnetism and Optics}, Spring 2007: Teaching Assistant

% Professional Activities
\sectiontitle{Professional Activities}
\vspace{-0.4cm}

Student of Google Summer of Code mentor in OpenCV from 2014 to 2017.

Mentor of Google Summer of Code mentor in OpenCV from 2018.

Member of the board of the official OpenCV foundation.

Creator of the Open Source Diffrentiable Computer Vision Library \href{www.kornia.org}{www.kornia.org}. Maintance and development from 2017.

Mantainer/developer of the \href{http://tiny-dnn.readthedocs.io}{tiny-dnn} library from 2016 to 2018.

Contributed in many Open Source projects such as OpenCV, kornia, tiny-dnn, Pytorch, ROS, Object Recognition Kitchen, OpenDroneMap and OpenSfM.

Participated in robotics competitions and campus such as the HUMABOT Robot Competition 2014 during the \textit{IEEE-RAS International Conference on Humanoid Robots} (Madrid, Spain) and the RoCKIn Camp 2014 organized by  La Sapienza University of Rome (Rome, Italy) within the company PAL Robotics.

Co-founder and member of the \textit{La Konfraria de la Vila del Pingui}. An Open Source community, organizing local events and workshops spreading the free software culture.


% Skills
\label{Skills}
\sectiontitle{Skills}
Competence: Computer Vision (local features, camera pose estimation), Deep Learning, Robotics, Programming, Open Source.\\
Programming Languages: Python, C++.\\
Programming Libraries: Pytorch, OpenCV, kornia, ROS. \\
Extra Interests: Git, Docker, Vim, CMake, Unix. \\
Languages: Catalan (native), Spanish (native), English  (professional)


% Software
\sectiontitle{Software}
{\em kornia}: Open Source Differentiable Computer Vision Library in PyTorch.

{\em tiny-dnn}: header only, dependency-free deep learning framework in C++11.


% Papers
%\sectiontitle{Journal Articles}
%\input{publications_journals.tex}

\sectiontitle{Papers in Reviewed Proceedings}
V.~Balntas, E.~Riba, D.~Ponsa, K.~Mikolajczyk, ``Learning local feature descriptors with triplets and shallow convolutional neural networks'', \textbf{BMVC}, 2016.

% Relevant Coursework
\label{Relevant Coursework}
\sectiontitle{Relevant Coursework}
- \textit{Object Detection} (Free online course, Universitat Autonoma de Barcelona. Fall 2015)\\
- \textit{Machine Learning} (Free online course, Stanford. Fall 2014)\\
- \textit{Introduction to Robot Operating System (ROS)} (Universitat Politecnica de Catalunya. Spring 2014)

\label{Reference}
\sectiontitle{References}
\textbf{Dr. Gary Bradski}\\
OSVF/OpenCV.org\\
garybradski@osvf.org

\textbf{Dr. Daniel Ponsa}\\
Computer Vision Center, Universitat Autonoma de Barcelona.\\
daniel@cvc.uab.es

\textbf{Dr. Francesc Moreno-Noguer}\\
Institut de Robotica i Informatica Industrial (CSIC-UPC)   
\\
fmoreno@iri.upc.edu

\textbf{Dr. Vincent Rabaud}\\
Google, Inc.\\
vincent.rabaud@gmail.com



\end{llist}

{\em Last update: \today}

\end{document}
