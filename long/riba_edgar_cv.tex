\documentclass{article}

\usepackage[english]{babel}
\usepackage{eurosym}
\usepackage{iflang}
\usepackage{resume}

\begin{document}

\selectlanguage{english}

\name{\bf EDGAR RIBA}

\begin{llist}
\label{Education}
\sectiontitle{Education}

\employer{UNIVERSITAT AUTONOMA DE BARCELONA} \location{Barcelona, Spain}
\dates{10/2015--now}
Ph.D. in Computer Science\\
Dissertation: \textit{Computer Vision Techniques in Very
High Resolution Remote Sensing}.\\
Advisor: Daniel Ponsa.\\
Research Topics: Deep Learning, Local Features, Camera Pose Estimation.

\employer{UNIVERSITAT POLITECNICA DE CATALUNYA} \location{Barcelona, Spain}
\dates{07/2012--06/2015}
M.S. in Automatic Control and Robotics.\\
Dissertation: \textit{Implementation of a 3D pose estimation algorithm}.\\
Advisors: Adrian Penate and Francesc Moreno-Noguer.

\employer{UNIVERSITAT POLITECNICA DE CATALUNYA} \location{Barcelona, Spain}
\dates{07/2008--06/2012}
B.S. in Geomatic and Surveying Engineering.


% Research Experience
\label{Work Experience}
\sectiontitle{Work Experience}
\vspace{-0.33cm}

\employer{COMPUTER VISION CENTER}\location{Barcelona, Spain}
\dates{10/2015--now}
PhD candidate, Advanced Driver Assistance Systems group
\vspace{-0.33cm}
\begin{itemize}
 \item working on deep learning algorithms to solve problems related to local features and camera pose estimation.
\end{itemize}

\employer{OPENCV}\location{online}
\dates{05/2016--08/2016}
Intern, Google Summer of Code
\vspace{-0.33cm}
\begin{itemize}
 \item integrated tiny-dnn to OpenCV contrib by adding a wrapper to the caffe converter.
 \item fixed bugs in tiny-dnn and developed several new features such as GPU support via OpenCL and NNPACK optimizations.
\end{itemize}

\employer{OPENCV}\location{online}
\dates{05/2015--08/2015}
Intern, Google Summer of Code
\vspace{-0.33cm}
\begin{itemize}
 \item developed the Structure From Motion module using a customized version of Libmv.
\end{itemize}

\employer{ALDEBARAN ROBOTICS}\location{Paris, France}
\dates{02/2015--06/2015}
Intern Software Engineer, Perception team
\vspace{-0.33cm}
\begin{itemize}
 \item designed and implemented an algorithm for people detection and tracking by sensor fusion using ROS, OpenCV and C++.
\end{itemize}

\employer{OPENCV}\location{online}
\dates{05/2014--08/2014}
Intern, Google Summer of Code
\vspace{-0.33cm}
\begin{itemize}
 \item designed and implemented a real time pose estimation algorithm for textured objects.
 \item implemented the PnP method: \textit{A direct least-squares} (DLS) in the calib3d module.
 \item contributed with a tutorial for the calib3d module.
\end{itemize}

\employer{INSTITUT DE ROBOTICA I INFORMATICA INDUSTRIAL (CSIC-UPC) }\location{Barcelona, Spain}
\dates{05/2014--08/2014}
Research Assistant, Perception and Manipulation group
\vspace{-0.33cm}
\begin{itemize}
 \item worked on my masters thesis in geometric computer vision.
\end{itemize}


% Teaching Experience
\label{Teaching Experience}
\sectiontitle{Teaching Experience}

\employer{UNIVERSITAT POLITECNICA DE CATALUNYA}\location{Barcelona, Spain}
310209, {\em Electromagnetism and Optics}, Fall 2007: Teaching Assistant\\
310209, {\em Electromagnetism and Optics}, Spring 2007: Teaching Assistant

\employer{UNIVERSITAT AUTONOMA DE BARCELONA}\location{Barcelona, Spain}
102708, {\em Software Engineering I}, Spring 2016: Teaching Assistant\\

% Professional Activities
\sectiontitle{Professional Activities}
\vspace{-0.4cm}

Google Summer of Code student for OpenCV from 2014 to 2016.

Mantainer/developer of the tiny-dnn library involved in the core development and design, bug fixes and code review.

Contributed in many Open Source projects such as OpenCV, ROS, Object Recognition Kitchen, OpenDroneMap and OpenSfM.

Participated in robotics competitions and campus such as the HUMABOT Robot Competition 2014 during the \textit{IEEE-RAS International Conference on Humanoid Robots} (Madrid, Spain) and the RoCKIn Camp 2014 organized by  La Sapienza University of Rome (Rome, Italy) within the company PAL Robotics.

Co-founder and member of the \textit{La Konfraria de la Vila del Pingui}. An Open Source community, organizing local events and workshops spreading the free software culture.


% Skills
\label{Skills}
\sectiontitle{Skills}
Competence: Comuter Vision (local features, camera pose estimation), Deep Learning, Robotics, Programming.\\
Programming Languages: C++, Python.\\
Programming Libraries: Tensorflow, tiny-dnn, OpenCV, ROS. \\
Extra Interests: Git, CMake, OpenCL, Octave/Matlab, Boost, Android. \\
Languages: Catalan (native), Spanish (native), English  (professional)


% Software
\sectiontitle{Software}
{\em tiny-dnn}: header only, dependency-free deep learning framework in C++11.


% Papers
%\sectiontitle{Journal Articles}
%\input{publications_journals.tex}

\sectiontitle{Papers in Reviewed Proceedings}
V.~Balntas, E.~Riba, D.~Ponsa, K.~Mikolajczyk, ``Learning local feature descriptors with triplets and shallow convolutional neural networks'', \textbf{BMVC}, 2016.

% Relevant Coursework
\label{Relevant Coursework}
\sectiontitle{Relevant Coursework}
- \textit{Object Detection} (Free online course, Universitat Autonoma de Barcelona. Fall 2015)\\
- \textit{Machine Learning} (Free online course, Stanford. Fall 2014)\\
- \textit{Introduction to Robot Operating System (ROS)} (Universitat Politecnica de Catalunya. Spring 2014)

\label{Reference}
\sectiontitle{References}
\textbf{Dr. Gary Bradski}\\
OSVF/OpenCV.org\\
garybradski@osvf.org

\textbf{Dr. Daniel Ponsa}\\
Computer Vision Center, Universitat Autonoma de Barcelona.\\
daniel@cvc.uab.es

\textbf{Dr. Francesc Moreno-Noguer}\\
Institut de Robotica i Informatica Industrial (CSIC-UPC)   
\\
fmoreno@iri.upc.edu

\textbf{Dr. Vincent Rabaud}\\
Google, Inc.\\
vincent.rabaud@gmail.com



\end{llist}

{\em Last update: \today}

\end{document}
